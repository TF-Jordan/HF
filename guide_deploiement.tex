%!TEX encoding = UTF-8 Unicode
%!TEX program = xelatex
\documentclass[11pt,a4paper]{report}

\usepackage[utf8]{inputenc}
\usepackage[T1]{fontenc}
\usepackage[french]{babel}
\usepackage[a4paper, top=2.5cm, bottom=2.5cm, left=2.5cm, right=2.5cm]{geometry}
\usepackage{graphicx}
\usepackage{float}
\usepackage{subcaption}
\usepackage{amsmath}
\usepackage{amssymb}
\usepackage{booktabs}
\usepackage{array}
\usepackage{tabularx}
\usepackage{longtable}
\usepackage{colortbl}
\usepackage{multirow}
\usepackage{enumitem}
\usepackage{tikz}
\usetikzlibrary{positioning, calc, shadows, shapes.geometric, arrows.meta, fit, backgrounds}
\usepackage{titlesec}
\usepackage{tocloft}
\usepackage{fancyhdr}
\usepackage{caption}
\usepackage{tcolorbox}
\tcbuselibrary{skins, breakable, listings}
\usepackage{fontawesome5}
\usepackage{hyperref}
\usepackage{xurl}
\usepackage{listings}
\Urlmuskip=0mu plus 1mu

% ── Couleurs ──
\definecolor{harmonyPrimary}{RGB}{37, 99, 235}
\definecolor{harmonyDark}{RGB}{30, 64, 175}
\definecolor{harmonyLight}{RGB}{219, 234, 254}
\definecolor{harmonyAccent}{RGB}{16, 185, 129}
\definecolor{harmonyWarm}{RGB}{245, 158, 11}
\definecolor{harmonyGray900}{RGB}{17, 24, 39}
\definecolor{harmonyGray700}{RGB}{55, 65, 81}
\definecolor{harmonyGray500}{RGB}{107, 114, 128}
\definecolor{harmonyGray300}{RGB}{209, 213, 219}
\definecolor{harmonyGray100}{RGB}{243, 244, 246}
\definecolor{harmonyWhite}{RGB}{255, 255, 255}
\definecolor{harmonySuccess}{RGB}{34, 197, 94}
\definecolor{harmonyWarning}{RGB}{234, 179, 8}
\definecolor{harmonyError}{RGB}{239, 68, 68}
\definecolor{codebg}{RGB}{30, 30, 46}
\definecolor{codegreen}{RGB}{166, 227, 161}
\definecolor{codeyellow}{RGB}{249, 226, 175}
\definecolor{codeblue}{RGB}{137, 180, 250}
\definecolor{codegray}{RGB}{147, 153, 178}

\hypersetup{
    colorlinks=true, linkcolor=harmonyDark,
    urlcolor=harmonyPrimary, citecolor=harmonyDark,
    bookmarksnumbered=true
}

% ── Listings YAML ──
\lstdefinelanguage{yaml}{
  keywords={name, on, push, pull_request, jobs, runs-on, steps, uses, with, run, if, env, secrets, needs, defaults, working-directory, distribution, java-version, flutter-version, channel, cache, path, true, false},
  sensitive=false,
  comment=[l]{\#},
  morestring=[b]',
  morestring=[b]"
}

\lstset{
  language=yaml,
  basicstyle=\ttfamily\small\color{harmonyWhite},
  backgroundcolor=\color{codebg},
  keywordstyle=\color{codeblue}\bfseries,
  stringstyle=\color{codegreen},
  commentstyle=\color{codegray}\itshape,
  identifierstyle=\color{codeyellow},
  frame=none,
  rulecolor=\color{harmonyGray300},
  breaklines=true,
  showstringspaces=false,
  tabsize=2,
  xleftmargin=12pt,
  xrightmargin=12pt,
  aboveskip=10pt,
  belowskip=10pt
}

% ── Code box ──
\newtcolorbox{codebox}[1][]{
    enhanced, breakable,
    colback=codebg, colframe=harmonyGray700,
    boxrule=1pt, arc=6pt,
    left=4pt, right=4pt, top=6pt, bottom=6pt,
    #1
}

% ── Boites ──
\newtcolorbox{infobox}[1][]{
    enhanced, breakable,
    colback=harmonyLight, colframe=harmonyPrimary,
    boxrule=0pt, leftrule=4pt, arc=0pt, outer arc=0pt,
    left=12pt, right=12pt, top=10pt, bottom=10pt, #1
}
\newtcolorbox{definitionbox}[1][]{
    enhanced, breakable,
    colback=harmonyGray100, colframe=harmonyGray300,
    fonttitle=\bfseries\sffamily, title=#1,
    boxrule=1pt, arc=6pt,
    left=12pt, right=12pt, top=8pt, bottom=8pt,
    attach boxed title to top left={yshift=-3mm, xshift=10pt},
    boxed title style={colback=harmonyPrimary, arc=3pt}
}
\newtcolorbox{processbox}[1][]{
    enhanced, breakable,
    colback=harmonyWhite, colframe=harmonyGray300,
    boxrule=1pt, arc=8pt,
    left=15pt, right=15pt, top=12pt, bottom=12pt,
    shadow={2pt}{-2pt}{0pt}{harmonyGray300}, #1
}
\newtcolorbox{alertbox}[1][]{
    enhanced, breakable,
    colback=harmonyWarm!10, colframe=harmonyWarm,
    boxrule=0pt, leftrule=4pt, arc=0pt,
    left=12pt, right=12pt, top=10pt, bottom=10pt, #1
}
\newtcolorbox{successbox}[1][]{
    enhanced, breakable,
    colback=harmonySuccess!10, colframe=harmonySuccess,
    boxrule=0pt, leftrule=4pt, arc=0pt,
    left=12pt, right=12pt, top=10pt, bottom=10pt, #1
}

% ── Chapitres ──
\titleformat{\chapter}[display]
    {\normalfont\huge\bfseries\sffamily}
    {\begin{tikzpicture}[remember picture, overlay]
        \fill[harmonyPrimary] (current page.north west) rectangle ([yshift=-3cm]current page.north east);
        \fill[harmonyAccent] ([yshift=-3cm]current page.north west) rectangle ([yshift=-3.15cm]current page.north east);
        \node[anchor=east, text=harmonyWhite, font=\fontsize{72}{80}\selectfont\bfseries\sffamily]
            at ([xshift=-2cm, yshift=-1.5cm]current page.north east) {\thechapter};
    \end{tikzpicture}}
    {0pt}{\vspace{2cm}\textcolor{harmonyDark}}
    [\vspace{0.5cm}{\color{harmonyGray300}\titlerule[2pt]}]

\titleformat{name=\chapter,numberless}[display]
    {\normalfont\huge\bfseries\sffamily}
    {\begin{tikzpicture}[remember picture, overlay]
        \fill[harmonyPrimary] (current page.north west) rectangle ([yshift=-2cm]current page.north east);
        \fill[harmonyAccent] ([yshift=-2cm]current page.north west) rectangle ([yshift=-2.1cm]current page.north east);
    \end{tikzpicture}}
    {0pt}{\vspace{1cm}\textcolor{harmonyDark}}
    [\vspace{0.5cm}{\color{harmonyGray300}\titlerule[2pt]}]
\titlespacing*{\chapter}{0pt}{0pt}{40pt}

\titleformat{\section}
    {\normalfont\Large\bfseries\sffamily\color{harmonyPrimary}}
    {\colorbox{harmonyPrimary}{\textcolor{harmonyWhite}{\hspace{8pt}\thesection\hspace{8pt}}}\hspace{12pt}}
    {0pt}{}[{\color{harmonyGray300}\titlerule[1pt]}]
\titlespacing*{\section}{0pt}{30pt}{15pt}

\titleformat{\subsection}
    {\normalfont\large\bfseries\sffamily\color{harmonyDark}}
    {\textcolor{harmonyPrimary}{\thesubsection}\hspace{10pt}}{0pt}{}
\titlespacing*{\subsection}{0pt}{20pt}{10pt}

\titleformat{\subsubsection}
    {\normalfont\normalsize\bfseries\sffamily\color{harmonyGray700}}
    {\textcolor{harmonyAccent}{\thesubsubsection}\hspace{8pt}}{0pt}{}

% ── Table des matieres ──
\renewcommand{\cfttoctitlefont}{\Huge\bfseries\sffamily\color{harmonyDark}}
\renewcommand{\cftchapfont}{\bfseries\sffamily\color{harmonyPrimary}}
\renewcommand{\cftchappagefont}{\bfseries\color{harmonyPrimary}}
\renewcommand{\cftsecfont}{\sffamily\color{harmonyGray900}}
\renewcommand{\cftsubsecfont}{\sffamily\color{harmonyGray500}}
\setlength{\cftbeforechapskip}{12pt}
\setlength{\cftbeforesecskip}{6pt}
\renewcommand{\cftloftitlefont}{\Huge\bfseries\sffamily\color{harmonyDark}}
\renewcommand{\cftfigfont}{\sffamily Figure }
\renewcommand{\cftlottitlefont}{\Huge\bfseries\sffamily\color{harmonyDark}}
\renewcommand{\cfttabfont}{\sffamily Tableau }

\captionsetup{
    labelfont={bf, sf, color=harmonyPrimary},
    textfont={sf, color=harmonyGray700},
    labelsep=period, justification=centering,
    font=small, skip=10pt
}

% ── En-tetes ──
\pagestyle{fancy}
\fancyhf{}
\fancyhead[L]{\sffamily\small\textcolor{harmonyGray500}{\leftmark}}
\fancyhead[R]{\sffamily\small\textcolor{harmonyPrimary}{\textbf{Harmony Gloves}}}
\fancyfoot[C]{\begin{tikzpicture}[baseline]
    \node[fill=harmonyPrimary, text=harmonyWhite, rounded corners=3pt, inner sep=5pt, font=\sffamily\small\bfseries] {\thepage};
\end{tikzpicture}}
\renewcommand{\headrulewidth}{0pt}
\renewcommand{\headrule}{\vspace{2pt}{\color{harmonyGray300}\hrule height 1pt}\vspace{1pt}{\color{harmonyAccent}\hrule height 2pt width 0.3\textwidth}}
\fancypagestyle{plain}{
    \fancyhf{}
    \fancyfoot[C]{\begin{tikzpicture}[baseline]
        \node[fill=harmonyPrimary, text=harmonyWhite, rounded corners=3pt, inner sep=5pt, font=\sffamily\small\bfseries] {\thepage};
    \end{tikzpicture}}
    \renewcommand{\headrulewidth}{0pt}
}

% ── Listes ──
\setlist[itemize,1]{label=\textcolor{harmonyPrimary}{\faChevronRight}, leftmargin=*, itemsep=6pt}
\setlist[itemize,2]{label=\textcolor{harmonyAccent}{$\circ$}, leftmargin=*, itemsep=4pt}
\setlist[enumerate,1]{label=\protect\circlednum{\arabic*}, leftmargin=*, itemsep=8pt}
\newcommand*\circlednum[1]{\tikz[baseline=(char.base)]{\node[shape=circle, fill=harmonyPrimary, text=harmonyWhite, inner sep=2pt, font=\sffamily\small\bfseries] (char) {#1};}}

% ── Commandes utilitaires ──
\newcommand{\concept}[1]{\textcolor{harmonyPrimary}{\textbf{#1}}}
\newcommand{\tech}[1]{\texttt{\textcolor{harmonyDark}{#1}}}
\newcommand{\flowto}{\textcolor{harmonyAccent}{\faArrowRight}}
\newcommand{\statusok}{\textcolor{harmonySuccess}{\faCheckCircle}}
\newcommand{\statuswarn}{\textcolor{harmonyWarning}{\faExclamationTriangle}}
\newcommand{\statusno}{\textcolor{harmonyError}{\faTimesCircle}}
\newcommand{\elegantsep}{\begin{center}\vspace{10pt}\textcolor{harmonyGray300}{\rule{0.2\textwidth}{0.5pt}\hspace{10pt}\textcolor{harmonyAccent}{\faDiamond}\hspace{10pt}\rule{0.2\textwidth}{0.5pt}}\vspace{10pt}\end{center}}

\begin{document}

% ════════════════════════════════════════════════════════════════
%                         PAGE DE TITRE
% ════════════════════════════════════════════════════════════════

\begin{titlepage}
    \begin{tikzpicture}[remember picture, overlay]
        \fill[harmonyGray900] (current page.south west) rectangle (current page.north east);
        \fill[harmonyPrimary, opacity=0.8] (current page.north west) -- ++(0,-8) -- ++(21,0) -- ++(0,8) -- cycle;
        \fill[harmonyDark, opacity=0.9] (current page.north west) -- ++(0,-6) -- ++(15,0) -- ++(6,-2) -- ++(0,8) -- cycle;
        \foreach \x/\y/\r/\o in {15/-5/3/0.1, 18/-8/2/0.15, 3/-12/4/0.08, 17/-18/2.5/0.12} {
            \fill[harmonyAccent, opacity=\o] (\x,-\y) circle (\r);
        }
        \draw[harmonyAccent, line width=2pt, opacity=0.5] (0,-10) -- (21,-10);
        \node[anchor=center] at (10.5,-4) {
            \begin{tikzpicture}
                \node[circle, fill=harmonyWhite, minimum size=3cm, opacity=0.1] {};
                \node[circle, fill=harmonyAccent, minimum size=2cm, opacity=0.2] {};
                \node[text=harmonyWhite, font=\fontsize{48}{50}\selectfont] {\faRocket};
            \end{tikzpicture}
        };
        \node[anchor=center, text=harmonyWhite] at (10.5,-9) {
            \begin{minipage}{16cm}
                \centering
                {\fontsize{14}{16}\selectfont\sffamily\textcolor{harmonyAccent}{GUIDE DE D\'EPLOIEMENT}}\\[0.8cm]
                {\fontsize{56}{60}\selectfont\bfseries\sffamily HARMONY}\\[0.2cm]
                {\fontsize{56}{60}\selectfont\bfseries\sffamily GLOVES}\\[1cm]
                {\color{harmonyGray300}\rule{8cm}{1pt}}\\[1cm]
                {\fontsize{18}{22}\selectfont\sffamily CI/CD avec GitHub Actions}
            \end{minipage}
        };
        \node[anchor=south, text=harmonyWhite] at (10.5,-26) {
            \begin{minipage}{16cm}
                \centering
                \begin{tikzpicture}
                    \node[fill=harmonyGray700, rounded corners=8pt, inner sep=15pt] {
                        \begin{tabular}{c@{\hspace{3cm}}c@{\hspace{3cm}}c}
                            \textcolor{harmonyAccent}{\faUsers} & \textcolor{harmonyAccent}{\faCalendarAlt} & \textcolor{harmonyAccent}{\faGithub} \\[5pt]
                            \textcolor{harmonyWhite}{\sffamily\small \'Equipe Harmony} &
                            \textcolor{harmonyWhite}{\sffamily\small F\'evrier 2026} &
                            \textcolor{harmonyWhite}{\sffamily\small GitHub Actions}
                        \end{tabular}
                    };
                \end{tikzpicture}
            \end{minipage}
        };
    \end{tikzpicture}
\end{titlepage}

\newpage
\tableofcontents
\newpage

% ════════════════════════════════════════════════════════════════
%             CHAPITRE 1 --- INTRODUCTION
% ════════════════════════════════════════════════════════════════

\chapter{Introduction au d\'eploiement}

\section{Objectif du guide}

Ce document d\'ecrit la proc\'edure compl\`ete pour \concept{automatiser la compilation, les tests et la distribution} de l'application mobile Harmony Gloves. Le pipeline CI/CD repose sur \concept{GitHub Actions}, un service d'int\'egration continue int\'egr\'e directement dans le d\'ep\^ot GitHub du projet.

\begin{infobox}
\textbf{\faInfoCircle\ Qu'est-ce que le CI/CD ?}\\[6pt]
\textbf{CI} (Continuous Integration) : chaque modification du code d\'eclenche automatiquement la compilation et les tests.\\
\textbf{CD} (Continuous Delivery) : l'APK compil\'e est automatiquement mis \`a disposition sous forme d'artifact t\'el\'echargeable.
\end{infobox}

\section{Vue d'ensemble du pipeline}

\begin{figure}[H]
\centering
\begin{tikzpicture}[
    node distance=1.2cm,
    step/.style={rectangle, rounded corners=8pt, draw=#1, fill=#1!12, text=harmonyGray900, minimum width=2.6cm, minimum height=1.4cm, font=\sffamily\small\bfseries, line width=1pt, align=center},
    arr/.style={-{Stealth[length=3mm]}, line width=1.5pt, harmonyAccent}
]
    \node[step=harmonyPrimary] (push) {\faCodeBranch\\[2pt] Push / PR};
    \node[step=harmonyWarm, right=1cm of push] (trigger) {\faGithub\\[2pt] GitHub\\Actions};
    \node[step=harmonyDark, right=1cm of trigger] (build) {\faHammer\\[2pt] Build\\APK};
    \node[step=harmonyAccent, right=1cm of build] (test) {\faVial\\[2pt] Tests};
    \node[step=harmonySuccess, right=1cm of test] (artifact) {\faDownload\\[2pt] Artifact\\APK};

    \draw[arr] (push) -- (trigger);
    \draw[arr] (trigger) -- (build);
    \draw[arr] (build) -- (test);
    \draw[arr] (test) -- (artifact);
\end{tikzpicture}
\caption{Pipeline CI/CD Harmony Gloves.}
\label{fig:pipeline}
\end{figure}

\section{Pr\'erequis}

\begin{processbox}
\begin{tabularx}{\textwidth}{>{\bfseries\sffamily}l X}
\faGithub\ Compte GitHub & Un compte GitHub avec acc\`es en \'ecriture au d\'ep\^ot Harmony. \\[8pt]
\faCodeBranch\ D\'ep\^ot & Le code source du projet pouss\'e sur un d\'ep\^ot GitHub. \\[8pt]
\faKey\ Secrets & (Optionnel) Keystore de signature APK configur\'e dans les secrets du d\'ep\^ot. \\[8pt]
\faLaptop\ Local & Git install\'e localement pour pousser les modifications.
\end{tabularx}
\end{processbox}


% ════════════════════════════════════════════════════════════════
%        CHAPITRE 2 --- ARCHITECTURE DU PIPELINE CI/CD
% ════════════════════════════════════════════════════════════════

\chapter{Architecture du pipeline CI/CD}

\section{Structure du d\'ep\^ot}

Le workflow GitHub Actions est d\'efini dans le fichier suivant :

Le diagramme ci-dessous illustre l'emplacement du fichier de workflow dans le d\'ep\^ot :


\begin{figure}[H]
\centering
\begin{tikzpicture}[
    folder/.style={rectangle, rounded corners=4pt, draw=harmonyPrimary, fill=harmonyLight, font=\sffamily\small\bfseries, text=harmonyDark, inner sep=6pt, minimum width=2.5cm},
    file/.style={rectangle, rounded corners=3pt, draw=harmonyGray300, fill=harmonyGray100, font=\ttfamily\small, text=harmonyGray700, inner sep=5pt},
    conn/.style={-, harmonyGray300, line width=0.8pt}
]
    \node[folder] (root) {\faFolder\ Harmony\_front};
    \node[folder, below left=1.2cm and 1cm of root] (gh) {\faFolder\ .github};
    \node[folder, below=0.8cm of gh] (wf) {\faFolder\ workflows};
    \node[file, below=0.8cm of wf] (yml) {\faFileCode\ flutter\_build.yml};

    \node[folder, below=1.2cm of root] (android) {\faFolder\ android};
    \node[folder, below right=1.2cm and 1cm of root] (lib) {\faFolder\ lib};
    \node[file, right=0.5cm of lib] (pubspec) {\faFileAlt\ pubspec.yaml};

    \draw[conn] (root) -- (gh);
    \draw[conn] (gh) -- (wf);
    \draw[conn] (wf) -- (yml);
    \draw[conn] (root) -- (android);
    \draw[conn] (root) -- (lib);
    \draw[conn] (root.east) -- ++(0.3,0) |- (pubspec);
\end{tikzpicture}
\caption{Emplacement du fichier de workflow dans le d\'ep\^ot.}
\label{fig:structure-repo}
\end{figure}

\section{Workflow actuel : \tech{flutter\_build.yml}}

Voici le workflow actuellement en place dans le projet :

\begin{codebox}
\begin{lstlisting}
name: Flutter Build

on:
  push:
  pull_request:

jobs:
  build-android:
    runs-on: ubuntu-latest
    defaults:
      run:
        working-directory: .
    steps:
      - name: Checkout
        uses: actions/checkout@v4
      - name: Setup Java
        uses: actions/setup-java@v4
        with:
          distribution: temurin
          java-version: "17"
      - name: Setup Flutter
        uses: subosito/flutter-action@v2
        with:
          flutter-version: "3.24.0"
          channel: stable
          cache: true
      - name: Setup Android SDK
        uses: android-actions/setup-android@v3
      - name: Install Android NDK
        run: sdkmanager "ndk;27.0.12077973"
      - name: Install dependencies
        run: flutter pub get
      - name: Build APK
        run: flutter build apk --release
      - name: Upload APK
        uses: actions/upload-artifact@v4
        with:
          name: app-release-apk
          path: build/app/outputs/flutter-apk/
                app-release.apk
\end{lstlisting}
\end{codebox}

\section{D\'ecomposition des \'etapes}

\begin{table}[H]
\centering
\rowcolors{2}{harmonyGray100}{harmonyWhite}
\begin{tabularx}{\textwidth}{c c >{\bfseries\sffamily\small}l >{\sffamily\small}X}
\toprule
\textbf{\textcolor{harmonyPrimary}{\#}} & \textbf{\textcolor{harmonyPrimary}{}} & \textbf{\textcolor{harmonyPrimary}{\'Etape}} & \textbf{\textcolor{harmonyPrimary}{Description}} \\
\midrule
\circlednum{1} & \textcolor{harmonyAccent}{\faCodeBranch} & Checkout & Clone le code source du d\'ep\^ot. \\
\addlinespace
\circlednum{2} & \textcolor{harmonyAccent}{\faJava} & Setup Java 17 & Installe le JDK Temurin 17 pour Gradle. \\
\addlinespace
\circlednum{3} & \textcolor{harmonyAccent}{\faMobileAlt} & Setup Flutter 3.24 & Installe le SDK Flutter avec cache. \\
\addlinespace
\circlednum{4} & \textcolor{harmonyAccent}{\faAndroid} & Setup Android SDK & Configure le SDK Android. \\
\addlinespace
\circlednum{5} & \textcolor{harmonyAccent}{\faMicrochip} & Install NDK 27 & Installe le NDK pour TFLite natif. \\
\addlinespace
\circlednum{6} & \textcolor{harmonyAccent}{\faDownload} & flutter pub get & T\'el\'echarge les d\'ependances Dart. \\
\addlinespace
\circlednum{7} & \textcolor{harmonyAccent}{\faHammer} & flutter build apk & Compile l'APK en mode release. \\
\addlinespace
\circlednum{8} & \textcolor{harmonyAccent}{\faCloudUploadAlt} & Upload Artifact & Archive l'APK comme artifact GitHub. \\
\bottomrule
\end{tabularx}
\caption{Les 8 \'etapes du workflow \tech{flutter\_build.yml}.}
\label{fig:etapes-workflow}
\end{table}

\begin{infobox}
\textbf{\faInfoCircle\ D\'eclencheurs :} Le workflow se lance automatiquement sur \textbf{tout push} (quelle que soit la branche) et sur \textbf{toute pull request}. Cela garantit que chaque modification est valid\'ee avant d'\^etre fusionn\'ee.
\end{infobox}


% ════════════════════════════════════════════════════════════════
%     CHAPITRE 3 --- CONFIGURATION DU D\'EP\^OT GITHUB
% ════════════════════════════════════════════════════════════════

\chapter{Configuration du d\'ep\^ot GitHub}

\section{Cr\'eation et initialisation du d\'ep\^ot}

\begin{enumerate}
    \item Cr\'eez un nouveau d\'ep\^ot sur GitHub (public ou priv\'e).
    \item Initialisez Git localement dans le dossier du projet :
\end{enumerate}

\begin{codebox}
\begin{lstlisting}[language=bash]
git init
git remote add origin https://github.com/
    votre-user/Harmony_front.git
git add .
git commit -m "Initial commit"
git push -u origin main
\end{lstlisting}
\end{codebox}

\section{Gestion des branches}

\begin{processbox}
Nous recommandons la strat\'egie de branches suivante :

\begin{tabularx}{\textwidth}{>{\bfseries\sffamily\color{harmonyPrimary}}l X}
\faCodeBranch\ main & Branche de production. Chaque commit d\'eclenche un build et produit un APK stable. \\[8pt]
\faCodeBranch\ develop & Branche de d\'eveloppement. Les fonctionnalit\'es en cours y sont int\'egr\'ees. \\[8pt]
\faCodeBranch\ feature/* & Branches de fonctionnalit\'es individuelles, fusionn\'ees dans \tech{develop} via Pull Request.
\end{tabularx}
\end{processbox}

\begin{figure}[H]
\centering
\begin{tikzpicture}[
    commit/.style={circle, fill=#1, minimum size=0.4cm, inner sep=0pt},
    branchlbl/.style={rectangle, rounded corners=4pt, fill=#1!15, draw=#1, font=\sffamily\scriptsize\bfseries, text=#1, inner sep=4pt}
]
    % main
    \node[branchlbl=harmonyPrimary] (lmain) at (0, 0) {main};
    \foreach \i in {1,...,5} {
        \node[commit=harmonyPrimary] (m\i) at (\i*1.5, 0) {};
    }
    \draw[harmonyPrimary, line width=2pt] (lmain) -- (m1);
    \foreach \i [count=\j from 2] in {1,...,4} {
        \draw[harmonyPrimary, line width=2pt] (m\i) -- (m\j);
    }

    % develop
    \node[branchlbl=harmonyAccent] (ldev) at (0, -1.5) {develop};
    \foreach \i in {1,...,4} {
        \pgfmathsetmacro{\xpos}{\i*1.5 + 0.5}
        \node[commit=harmonyAccent] (d\i) at (\xpos, -1.5) {};
    }
    \draw[harmonyAccent, line width=1.5pt] (m1) -- (d1);
    \foreach \i [count=\j from 2] in {1,...,3} {
        \draw[harmonyAccent, line width=1.5pt] (d\i) -- (d\j);
    }
    \draw[harmonyAccent, line width=1.5pt, dashed] (d2) -- (m3);
    \draw[harmonyAccent, line width=1.5pt, dashed] (d4) -- (m5);

    % feature
    \node[branchlbl=harmonyWarm] (lfeat) at (0, -3) {feature/tts};
    \node[commit=harmonyWarm] (f1) at (3, -3) {};
    \node[commit=harmonyWarm] (f2) at (4.5, -3) {};
    \draw[harmonyWarm, line width=1pt] (d1) -- (f1);
    \draw[harmonyWarm, line width=1pt] (f1) -- (f2);
    \draw[harmonyWarm, line width=1pt, dashed] (f2) -- (d3);
\end{tikzpicture}
\caption{Strat\'egie de branches recommand\'ee (Git Flow simplifi\'e).}
\label{fig:gitflow}
\end{figure}

\section{Gestion des secrets GitHub}

Pour la signature de l'APK en production, des secrets doivent \^etre configur\'es :

\begin{enumerate}
    \item Naviguez vers \concept{Settings} \flowto\ \concept{Secrets and variables} \flowto\ \concept{Actions} dans votre d\'ep\^ot GitHub.
    \item Ajoutez les secrets suivants :
\end{enumerate}

\begin{table}[H]
\centering
\rowcolors{2}{harmonyGray100}{harmonyWhite}
\begin{tabularx}{\textwidth}{>{\bfseries\sffamily\small}l >{\sffamily\small}X}
\toprule
\textbf{\textcolor{harmonyPrimary}{Secret}} & \textbf{\textcolor{harmonyPrimary}{Description}} \\
\midrule
KEYSTORE\_BASE64 & Le keystore encod\'e en base64 (voir chapitre~\ref{chap:signature}). \\
KEY\_ALIAS & L'alias de la cl\'e dans le keystore (ex : \tech{harmony-key}). \\
KEY\_PASSWORD & Le mot de passe de la cl\'e. \\
STORE\_PASSWORD & Le mot de passe du keystore. \\
\bottomrule
\end{tabularx}
\caption{Secrets GitHub requis pour la signature APK.}
\label{tab:secrets}
\end{table}

\begin{alertbox}
\textbf{\faLock\ S\'ecurit\'e :} Les secrets GitHub sont chiffr\'es et ne sont jamais affich\'es dans les logs. Ne les partagez jamais dans le code source ou dans les messages de commit.
\end{alertbox}

% ════════════════════════════════════════════════════════════════
%     CHAPITRE 4 --- SIGNATURE DE L'APK
% ════════════════════════════════════════════════════════════════

\chapter{Signature de l'APK pour la production}
\label{chap:signature}

\section{G\'en\'eration du keystore}

Un \concept{keystore} est un fichier contenant la cl\'e cryptographique utilis\'ee pour signer l'APK. Il garantit l'authenticit\'e de l'application.

\begin{codebox}
\begin{lstlisting}[language=bash]
keytool -genkey -v \
  -keystore harmony-release.jks \
  -keyalg RSA -keysize 2048 \
  -validity 10000 \
  -alias harmony-key
\end{lstlisting}
\end{codebox}

\begin{alertbox}
\textbf{\faExclamationTriangle\ Important :} Conservez le fichier \tech{harmony-release.jks} et les mots de passe en lieu s\^ur. La perte du keystore emp\^eche toute mise \`a jour de l'application sur le Play Store ou la redistribution sign\'ee.
\end{alertbox}

\section{Configuration locale : \tech{key.properties}}

Cr\'eez le fichier \tech{android/key.properties} (ne pas le commiter dans Git) :

\begin{codebox}
\begin{lstlisting}
storePassword=votre_mot_de_passe_store
keyPassword=votre_mot_de_passe_cle
keyAlias=harmony-key
storeFile=../harmony-release.jks
\end{lstlisting}
\end{codebox}

\begin{successbox}
\statusok\ Ajoutez \tech{android/key.properties} et \tech{*.jks} dans votre fichier \tech{.gitignore} pour \'eviter toute fuite de secrets.
\end{successbox}

\section{Encodage du keystore en base64}

Pour stocker le keystore dans les secrets GitHub :

\begin{codebox}
\begin{lstlisting}[language=bash]
base64 -w 0 harmony-release.jks > keystore_b64.txt
\end{lstlisting}
\end{codebox}

Copiez le contenu de \tech{keystore\_b64.txt} dans le secret GitHub \tech{KEYSTORE\_BASE64}.

\section{Workflow avec signature}

Voici le workflow enrichi pour signer automatiquement l'APK :

\begin{codebox}
\begin{lstlisting}
name: Flutter Build & Sign

on:
  push:
    branches: [main]
  pull_request:
    branches: [main]

jobs:
  build-android:
    runs-on: ubuntu-latest
    steps:
      - name: Checkout
        uses: actions/checkout@v4

      - name: Setup Java
        uses: actions/setup-java@v4
        with:
          distribution: temurin
          java-version: "17"

      - name: Setup Flutter
        uses: subosito/flutter-action@v2
        with:
          flutter-version: "3.24.0"
          channel: stable
          cache: true

      - name: Setup Android SDK
        uses: android-actions/setup-android@v3

      - name: Install Android NDK
        run: sdkmanager "ndk;27.0.12077973"

      - name: Decode Keystore
        run: |
          echo "${{ secrets.KEYSTORE_BASE64 }}" \
            | base64 -d > android/keystore.jks

      - name: Create key.properties
        run: |
          cat > android/key.properties << EOF
          storePassword=${{
              secrets.STORE_PASSWORD }}
          keyPassword=${{
              secrets.KEY_PASSWORD }}
          keyAlias=${{
              secrets.KEY_ALIAS }}
          storeFile=keystore.jks
          EOF

      - name: Install dependencies
        run: flutter pub get

      - name: Build signed APK
        run: flutter build apk --release

      - name: Upload APK
        uses: actions/upload-artifact@v4
        with:
          name: harmony-signed-apk
          path: build/app/outputs/
                flutter-apk/app-release.apk
\end{lstlisting}
\end{codebox}

\begin{figure}[H]
\centering
\begin{tikzpicture}[
    step/.style={rectangle, rounded corners=6pt, draw=#1, fill=#1!10, minimum width=2.2cm, minimum height=1.2cm, font=\sffamily\scriptsize\bfseries, text=harmonyGray900, align=center, line width=0.8pt},
    arr/.style={-{Stealth[length=2.5mm]}, line width=1.2pt, harmonyAccent}
]
    \node[step=harmonyPrimary] (s1) {\faLock\\Secrets};
    \node[step=harmonyWarm, right=0.6cm of s1] (s2) {\faKey\\D\'ecoder\\Keystore};
    \node[step=harmonyDark, right=0.6cm of s2] (s3) {\faFileAlt\\Cr\'eer\\key.props};
    \node[step=harmonyPrimary, right=0.6cm of s3] (s4) {\faHammer\\Build\\APK};
    \node[step=harmonySuccess, right=0.6cm of s4] (s5) {\faShieldAlt\\APK\\sign\'e};

    \draw[arr] (s1) -- (s2);
    \draw[arr] (s2) -- (s3);
    \draw[arr] (s3) -- (s4);
    \draw[arr] (s4) -- (s5);
\end{tikzpicture}
\caption{Flux de signature automatique de l'APK dans GitHub Actions.}
\label{fig:signature-flow}
\end{figure}


% ════════════════════════════════════════════════════════════════
%     CHAPITRE 5 --- AM\'ELIORATIONS DU WORKFLOW
% ════════════════════════════════════════════════════════════════

\chapter{Am\'eliorations du workflow}

\section{Analyse statique du code}

Ajoutez une \'etape d'analyse pour d\'etecter les erreurs et les mauvaises pratiques :

\begin{codebox}
\begin{lstlisting}
      - name: Analyze code
        run: flutter analyze --no-fatal-infos
\end{lstlisting}
\end{codebox}

\begin{infobox}
\textbf{\faInfoCircle\ Conseil :} L'option \tech{-{}-no-fatal-infos} permet de ne pas bloquer le build pour de simples informations. Les warnings et erreurs restent bloquants.
\end{infobox}

\section{Tests unitaires}

\begin{codebox}
\begin{lstlisting}
      - name: Run tests
        run: flutter test
\end{lstlisting}
\end{codebox}

\section{Versionnage automatique}

Automatisez le num\'ero de build en utilisant le num\'ero de run GitHub :

\begin{codebox}
\begin{lstlisting}
      - name: Build APK with auto version
        run: |
          flutter build apk --release \
            --build-number=${{ github.run_number }}
\end{lstlisting}
\end{codebox}

\section{Cr\'eation automatique de Releases GitHub}

Pour publier automatiquement une release avec l'APK attach\'e lorsqu'un tag est pouss\'e :

\begin{codebox}
\begin{lstlisting}
  release:
    needs: build-android
    if: startsWith(github.ref, 'refs/tags/v')
    runs-on: ubuntu-latest
    steps:
      - name: Download APK
        uses: actions/download-artifact@v4
        with:
          name: harmony-signed-apk

      - name: Create Release
        uses: softprops/action-gh-release@v2
        with:
          files: app-release.apk
          generate_release_notes: true
\end{lstlisting}
\end{codebox}

\begin{successbox}
\statusok\ Pour d\'eclencher une release, il suffit de cr\'eer un tag :
\tech{git tag v1.0.0 \&\& git push origin v1.0.0}
\end{successbox}

\section{Workflow complet am\'elior\'e}

\begin{figure}[H]
\centering
\begin{tikzpicture}[
    step/.style={rectangle, rounded corners=6pt, draw=#1, fill=#1!10, minimum width=2.2cm, minimum height=1cm, font=\sffamily\scriptsize\bfseries, text=harmonyGray900, align=center, line width=0.8pt},
    arr/.style={-{Stealth[length=2.5mm]}, line width=1pt, harmonyAccent},
    grp/.style={rectangle, rounded corners=8pt, draw=#1, dashed, inner sep=8pt, line width=1pt}
]
    \node[step=harmonyPrimary] (checkout) {\faCodeBranch\\Checkout};
    \node[step=harmonyDark, right=0.5cm of checkout] (setup) {\faCogs\\Setup\\Env};
    \node[step=harmonyWarm, right=0.5cm of setup] (analyze) {\faSearch\\Analyze};
    \node[step=harmonyAccent, right=0.5cm of analyze] (test) {\faVial\\Test};
    \node[step=harmonyPrimary, right=0.5cm of test] (build) {\faHammer\\Build};
    \node[step=harmonySuccess, right=0.5cm of build] (upload) {\faUpload\\Upload};

    \node[step=harmonyError, below=1cm of upload] (release) {\faTag\\Release};

    \draw[arr] (checkout) -- (setup);
    \draw[arr] (setup) -- (analyze);
    \draw[arr] (analyze) -- (test);
    \draw[arr] (test) -- (build);
    \draw[arr] (build) -- (upload);
    \draw[arr, dashed, harmonyError] (upload) -- node[right, font=\sffamily\tiny, text=harmonyGray500] {si tag} (release);

    \node[grp=harmonyGray300, fit=(checkout)(setup)(analyze)(test)(build)(upload), label={[font=\sffamily\scriptsize\bfseries, text=harmonyGray500]above:Job: build-android}] {};
\end{tikzpicture}
\caption{Pipeline CI/CD am\'elior\'e avec analyse, tests et releases automatiques.}
\label{fig:pipeline-ameliore}
\end{figure}


% ════════════════════════════════════════════════════════════════
%     CHAPITRE 6 --- DISTRIBUTION DE L'APK
% ════════════════════════════════════════════════════════════════

\chapter{Distribution de l'APK}

\section{R\'ecup\'eration depuis GitHub Actions}

\begin{enumerate}
    \item Acc\'edez \`a l'onglet \concept{Actions} de votre d\'ep\^ot GitHub.
    \item Cliquez sur le dernier \concept{workflow run} r\'eussi (indicateur vert \statusok).
    \item Dans la section \concept{Artifacts}, t\'el\'echargez \tech{harmony-signed-apk}.
    \item D\'ecompressez le fichier ZIP pour obtenir \tech{app-release.apk}.
\end{enumerate}

\begin{figure}[H]
\centering
\begin{tikzpicture}[
    box/.style={rectangle, rounded corners=8pt, draw=harmonyGray300, fill=harmonyWhite, minimum width=9cm, minimum height=1cm, font=\sffamily\small, text=harmonyGray900, line width=0.8pt}
]
    \node[box] (b1) at (0,0) {\faGithub\ \textbf{GitHub} \flowto\ Onglet \textbf{Actions}};
    \node[box] (b2) at (0,-1.5) {\statusok\ \textbf{Flutter Build} --- Run \#42 --- \textcolor{harmonySuccess}{Success}};
    \node[box, draw=harmonyPrimary, fill=harmonyLight] (b3) at (0,-3) {\faDownload\ \textbf{Artifacts :} harmony-signed-apk (15.2 MB)};

    \draw[-{Stealth}, harmonyAccent, line width=1pt] (b1) -- (b2);
    \draw[-{Stealth}, harmonyAccent, line width=1pt] (b2) -- (b3);
\end{tikzpicture}
\caption{T\'el\'echargement de l'APK depuis l'interface GitHub Actions.}
\label{fig:download-artifact}
\end{figure}

\section{Distribution via GitHub Releases}

Si le workflow de release automatique est configur\'e (chapitre 5), chaque tag cr\'ee une page de release publique :

\begin{processbox}
\begin{enumerate}
    \item Cr\'eez un tag de version : \tech{git tag v1.2.0 \&\& git push origin v1.2.0}
    \item GitHub Actions compile l'APK et cr\'ee automatiquement une \concept{Release}.
    \item Partagez le lien de la release avec les utilisateurs :
    \tech{https://github.com/user/Harmony\_front/releases/latest}
\end{enumerate}
\end{processbox}

\section{Distribution manuelle}

Pour distribuer l'APK sans passer par le Play Store :

\begin{itemize}
    \item \faShare\ \textbf{Partage direct} : Envoyez le fichier \tech{app-release.apk} par email, Bluetooth, WhatsApp ou cl\'e USB.
    \item \faQrcode\ \textbf{QR Code} : H\'ebergez l'APK sur un serveur web et g\'en\'erez un QR code pointant vers l'URL de t\'el\'echargement.
    \item \faCloud\ \textbf{Cloud} : Uploadez l'APK sur Google Drive, Dropbox ou un service \'equivalent et partagez le lien.
\end{itemize}

\section{Publication sur le Google Play Store (optionnel)}

\begin{alertbox}
\textbf{\faExclamationTriangle\ Pr\'erequis :} Un compte Google Play Developer (frais uniques de 25\$) et l'APK sign\'e avec un keystore de production.
\end{alertbox}

\begin{enumerate}
    \item Cr\'eez une fiche d'application sur la \concept{Google Play Console}.
    \item Configurez les informations de l'application (description, captures d'\'ecran, ic\^one).
    \item Uploadez l'APK sign\'e dans la section \concept{Production}.
    \item Soumettez pour examen par Google (d\'elai : 1 \`a 7 jours).
\end{enumerate}


% ════════════════════════════════════════════════════════════════
%     CHAPITRE 7 --- D\'EPANNAGE CI/CD
% ════════════════════════════════════════════════════════════════

\chapter{D\'epannage CI/CD}

\section{Erreurs fr\'equentes}

\begin{table}[H]
\centering
\rowcolors{2}{harmonyGray100}{harmonyWhite}
\begin{tabularx}{\textwidth}{>{\sffamily\small}p{3.2cm} >{\sffamily\small}X >{\sffamily\small}X}
\toprule
\textbf{\textcolor{harmonyPrimary}{Erreur}} & \textbf{\textcolor{harmonyPrimary}{Cause}} & \textbf{\textcolor{harmonyPrimary}{Solution}} \\
\midrule
NDK not found & Version du NDK non install\'ee ou incompatible & V\'erifiez la version dans le workflow : \tech{ndk;27.0.12077973} \\
\addlinespace
Gradle build failed & Version Java incompatible & Utilisez Java 17 (Temurin) comme dans le workflow \\
\addlinespace
Flutter version mismatch & SDK local diff\'erent du CI & Alignez la version dans \tech{flutter-version} du workflow \\
\addlinespace
Keystore not found & Secret mal configur\'e ou \'etape manquante & V\'erifiez les noms des secrets et l'\'etape \tech{Decode Keystore} \\
\addlinespace
Cache miss & Premi\`ere ex\'ecution ou cache expir\'e & Normal au premier run, les suivants seront plus rapides \\
\addlinespace
pub get failed & D\'ependance introuvable ou version incompatible & V\'erifiez \tech{pubspec.yaml} et les contraintes de version \\
\bottomrule
\end{tabularx}
\caption{Erreurs CI/CD fr\'equentes et solutions.}
\label{tab:depannage-cicd}
\end{table}

\section{Lecture des logs GitHub Actions}

\begin{enumerate}
    \item Acc\'edez \`a l'onglet \concept{Actions} du d\'ep\^ot.
    \item Cliquez sur le run \'echou\'e (\statusno\ en rouge).
    \item D\'epliez l'\'etape en \'echec pour voir les logs d\'etaill\'es.
    \item Cherchez les lignes marqu\'ees \textbf{Error} ou \textbf{FAILURE} pour identifier le probl\`eme.
\end{enumerate}

\begin{infobox}
\textbf{\faInfoCircle\ Astuce :} Utilisez la fonction \concept{Search logs} (loupe) en haut des logs pour rechercher rapidement un mot-cl\'e d'erreur sp\'ecifique.
\end{infobox}

\section{Optimisation des temps de build}

\begin{processbox}
\textbf{\textcolor{harmonyPrimary}{\faTachometerAlt\ Strat\'egies d'optimisation :}}

\begin{tabularx}{\textwidth}{>{\bfseries\sffamily}l X >{\sffamily}c}
\toprule
Strat\'egie & Description & Gain estim\'e \\
\midrule
Cache Flutter & D\'ej\`a activ\'e via \tech{cache: true} dans le workflow & $\sim$2 min \\
Cache Gradle & Ajouter \tech{actions/cache} pour \tech{\textasciitilde/.gradle} & $\sim$1 min \\
Filtrer les branches & Ne builder que \tech{main} et \tech{develop} & \'Evite les builds inutiles \\
Skip CI & Ajouter \tech{[skip ci]} dans le message de commit & Build non d\'eclench\'e \\
\bottomrule
\end{tabularx}
\end{processbox}

\begin{codebox}
\begin{lstlisting}
      - name: Cache Gradle
        uses: actions/cache@v4
        with:
          path: |
            ~/.gradle/caches
            ~/.gradle/wrapper
          key: gradle-${{ hashFiles(
            '**/*.gradle*') }}
          restore-keys: gradle-
\end{lstlisting}
\end{codebox}

\elegantsep

\begin{successbox}
\statusok\ \textbf{F\'elicitations !} Votre pipeline CI/CD est d\'esormais pleinement op\'erationnel. Chaque push sur \tech{main} produit automatiquement un APK sign\'e, test\'e et disponible en t\'el\'echargement via GitHub Actions ou GitHub Releases.
\end{successbox}


\end{document}
