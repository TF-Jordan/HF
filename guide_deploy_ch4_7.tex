% ════════════════════════════════════════════════════════════════
%     CHAPITRE 4 --- SIGNATURE DE L'APK
% ════════════════════════════════════════════════════════════════

\chapter{Signature de l'APK pour la production}
\label{chap:signature}

\section{G\'en\'eration du keystore}

Un \concept{keystore} est un fichier contenant la cl\'e cryptographique utilis\'ee pour signer l'APK. Il garantit l'authenticit\'e de l'application.

\begin{codebox}
\begin{lstlisting}[language=bash]
keytool -genkey -v \
  -keystore harmony-release.jks \
  -keyalg RSA -keysize 2048 \
  -validity 10000 \
  -alias harmony-key
\end{lstlisting}
\end{codebox}

\begin{alertbox}
\textbf{\faExclamationTriangle\ Important :} Conservez le fichier \tech{harmony-release.jks} et les mots de passe en lieu s\^ur. La perte du keystore emp\^eche toute mise \`a jour de l'application sur le Play Store ou la redistribution sign\'ee.
\end{alertbox}

\section{Configuration locale : \tech{key.properties}}

Cr\'eez le fichier \tech{android/key.properties} (ne pas le commiter dans Git) :

\begin{codebox}
\begin{lstlisting}
storePassword=votre_mot_de_passe_store
keyPassword=votre_mot_de_passe_cle
keyAlias=harmony-key
storeFile=../harmony-release.jks
\end{lstlisting}
\end{codebox}

\begin{successbox}
\statusok\ Ajoutez \tech{android/key.properties} et \tech{*.jks} dans votre fichier \tech{.gitignore} pour \'eviter toute fuite de secrets.
\end{successbox}

\section{Encodage du keystore en base64}

Pour stocker le keystore dans les secrets GitHub :

\begin{codebox}
\begin{lstlisting}[language=bash]
base64 -w 0 harmony-release.jks > keystore_b64.txt
\end{lstlisting}
\end{codebox}

Copiez le contenu de \tech{keystore\_b64.txt} dans le secret GitHub \tech{KEYSTORE\_BASE64}.

\section{Workflow avec signature}

Voici le workflow enrichi pour signer automatiquement l'APK :

\begin{codebox}
\begin{lstlisting}
name: Flutter Build & Sign

on:
  push:
    branches: [main]
  pull_request:
    branches: [main]

jobs:
  build-android:
    runs-on: ubuntu-latest
    steps:
      - name: Checkout
        uses: actions/checkout@v4

      - name: Setup Java
        uses: actions/setup-java@v4
        with:
          distribution: temurin
          java-version: "17"

      - name: Setup Flutter
        uses: subosito/flutter-action@v2
        with:
          flutter-version: "3.24.0"
          channel: stable
          cache: true

      - name: Setup Android SDK
        uses: android-actions/setup-android@v3

      - name: Install Android NDK
        run: sdkmanager "ndk;27.0.12077973"

      - name: Decode Keystore
        run: |
          echo "${{ secrets.KEYSTORE_BASE64 }}" \
            | base64 -d > android/keystore.jks

      - name: Create key.properties
        run: |
          cat > android/key.properties << EOF
          storePassword=${{
              secrets.STORE_PASSWORD }}
          keyPassword=${{
              secrets.KEY_PASSWORD }}
          keyAlias=${{
              secrets.KEY_ALIAS }}
          storeFile=keystore.jks
          EOF

      - name: Install dependencies
        run: flutter pub get

      - name: Build signed APK
        run: flutter build apk --release

      - name: Upload APK
        uses: actions/upload-artifact@v4
        with:
          name: harmony-signed-apk
          path: build/app/outputs/
                flutter-apk/app-release.apk
\end{lstlisting}
\end{codebox}

\begin{figure}[H]
\centering
\begin{tikzpicture}[
    step/.style={rectangle, rounded corners=6pt, draw=#1, fill=#1!10, minimum width=2.2cm, minimum height=1.2cm, font=\sffamily\scriptsize\bfseries, text=harmonyGray900, align=center, line width=0.8pt},
    arr/.style={-{Stealth[length=2.5mm]}, line width=1.2pt, harmonyAccent}
]
    \node[step=harmonyPrimary] (s1) {\faLock\\Secrets};
    \node[step=harmonyWarm, right=0.6cm of s1] (s2) {\faKey\\D\'ecoder\\Keystore};
    \node[step=harmonyDark, right=0.6cm of s2] (s3) {\faFileAlt\\Cr\'eer\\key.props};
    \node[step=harmonyPrimary, right=0.6cm of s3] (s4) {\faHammer\\Build\\APK};
    \node[step=harmonySuccess, right=0.6cm of s4] (s5) {\faShieldAlt\\APK\\sign\'e};

    \draw[arr] (s1) -- (s2);
    \draw[arr] (s2) -- (s3);
    \draw[arr] (s3) -- (s4);
    \draw[arr] (s4) -- (s5);
\end{tikzpicture}
\caption{Flux de signature automatique de l'APK dans GitHub Actions.}
\label{fig:signature-flow}
\end{figure}


% ════════════════════════════════════════════════════════════════
%     CHAPITRE 5 --- AM\'ELIORATIONS DU WORKFLOW
% ════════════════════════════════════════════════════════════════

\chapter{Am\'eliorations du workflow}

\section{Analyse statique du code}

Ajoutez une \'etape d'analyse pour d\'etecter les erreurs et les mauvaises pratiques :

\begin{codebox}
\begin{lstlisting}
      - name: Analyze code
        run: flutter analyze --no-fatal-infos
\end{lstlisting}
\end{codebox}

\begin{infobox}
\textbf{\faInfoCircle\ Conseil :} L'option \tech{-{}-no-fatal-infos} permet de ne pas bloquer le build pour de simples informations. Les warnings et erreurs restent bloquants.
\end{infobox}

\section{Tests unitaires}

\begin{codebox}
\begin{lstlisting}
      - name: Run tests
        run: flutter test
\end{lstlisting}
\end{codebox}

\section{Versionnage automatique}

Automatisez le num\'ero de build en utilisant le num\'ero de run GitHub :

\begin{codebox}
\begin{lstlisting}
      - name: Build APK with auto version
        run: |
          flutter build apk --release \
            --build-number=${{ github.run_number }}
\end{lstlisting}
\end{codebox}

\section{Cr\'eation automatique de Releases GitHub}

Pour publier automatiquement une release avec l'APK attach\'e lorsqu'un tag est pouss\'e :

\begin{codebox}
\begin{lstlisting}
  release:
    needs: build-android
    if: startsWith(github.ref, 'refs/tags/v')
    runs-on: ubuntu-latest
    steps:
      - name: Download APK
        uses: actions/download-artifact@v4
        with:
          name: harmony-signed-apk

      - name: Create Release
        uses: softprops/action-gh-release@v2
        with:
          files: app-release.apk
          generate_release_notes: true
\end{lstlisting}
\end{codebox}

\begin{successbox}
\statusok\ Pour d\'eclencher une release, il suffit de cr\'eer un tag :
\tech{git tag v1.0.0 \&\& git push origin v1.0.0}
\end{successbox}

\section{Workflow complet am\'elior\'e}

\begin{figure}[H]
\centering
\begin{tikzpicture}[
    step/.style={rectangle, rounded corners=6pt, draw=#1, fill=#1!10, minimum width=2.2cm, minimum height=1cm, font=\sffamily\scriptsize\bfseries, text=harmonyGray900, align=center, line width=0.8pt},
    arr/.style={-{Stealth[length=2.5mm]}, line width=1pt, harmonyAccent},
    grp/.style={rectangle, rounded corners=8pt, draw=#1, dashed, inner sep=8pt, line width=1pt}
]
    \node[step=harmonyPrimary] (checkout) {\faCodeBranch\\Checkout};
    \node[step=harmonyDark, right=0.5cm of checkout] (setup) {\faCogs\\Setup\\Env};
    \node[step=harmonyWarm, right=0.5cm of setup] (analyze) {\faSearch\\Analyze};
    \node[step=harmonyAccent, right=0.5cm of analyze] (test) {\faVial\\Test};
    \node[step=harmonyPrimary, right=0.5cm of test] (build) {\faHammer\\Build};
    \node[step=harmonySuccess, right=0.5cm of build] (upload) {\faUpload\\Upload};

    \node[step=harmonyError, below=1cm of upload] (release) {\faTag\\Release};

    \draw[arr] (checkout) -- (setup);
    \draw[arr] (setup) -- (analyze);
    \draw[arr] (analyze) -- (test);
    \draw[arr] (test) -- (build);
    \draw[arr] (build) -- (upload);
    \draw[arr, dashed, harmonyError] (upload) -- node[right, font=\sffamily\tiny, text=harmonyGray500] {si tag} (release);

    \node[grp=harmonyGray300, fit=(checkout)(setup)(analyze)(test)(build)(upload), label={[font=\sffamily\scriptsize\bfseries, text=harmonyGray500]above:Job: build-android}] {};
\end{tikzpicture}
\caption{Pipeline CI/CD am\'elior\'e avec analyse, tests et releases automatiques.}
\label{fig:pipeline-ameliore}
\end{figure}


% ════════════════════════════════════════════════════════════════
%     CHAPITRE 6 --- DISTRIBUTION DE L'APK
% ════════════════════════════════════════════════════════════════

\chapter{Distribution de l'APK}

\section{R\'ecup\'eration depuis GitHub Actions}

\begin{enumerate}
    \item Acc\'edez \`a l'onglet \concept{Actions} de votre d\'ep\^ot GitHub.
    \item Cliquez sur le dernier \concept{workflow run} r\'eussi (indicateur vert \statusok).
    \item Dans la section \concept{Artifacts}, t\'el\'echargez \tech{harmony-signed-apk}.
    \item D\'ecompressez le fichier ZIP pour obtenir \tech{app-release.apk}.
\end{enumerate}

\begin{figure}[H]
\centering
\begin{tikzpicture}[
    box/.style={rectangle, rounded corners=8pt, draw=harmonyGray300, fill=harmonyWhite, minimum width=9cm, minimum height=1cm, font=\sffamily\small, text=harmonyGray900, line width=0.8pt}
]
    \node[box] (b1) at (0,0) {\faGithub\ \textbf{GitHub} \flowto\ Onglet \textbf{Actions}};
    \node[box] (b2) at (0,-1.5) {\statusok\ \textbf{Flutter Build} --- Run \#42 --- \textcolor{harmonySuccess}{Success}};
    \node[box, draw=harmonyPrimary, fill=harmonyLight] (b3) at (0,-3) {\faDownload\ \textbf{Artifacts :} harmony-signed-apk (15.2 MB)};

    \draw[-{Stealth}, harmonyAccent, line width=1pt] (b1) -- (b2);
    \draw[-{Stealth}, harmonyAccent, line width=1pt] (b2) -- (b3);
\end{tikzpicture}
\caption{T\'el\'echargement de l'APK depuis l'interface GitHub Actions.}
\label{fig:download-artifact}
\end{figure}

\section{Distribution via GitHub Releases}

Si le workflow de release automatique est configur\'e (chapitre 5), chaque tag cr\'ee une page de release publique :

\begin{processbox}
\begin{enumerate}
    \item Cr\'eez un tag de version : \tech{git tag v1.2.0 \&\& git push origin v1.2.0}
    \item GitHub Actions compile l'APK et cr\'ee automatiquement une \concept{Release}.
    \item Partagez le lien de la release avec les utilisateurs :
    \tech{https://github.com/user/Harmony\_front/releases/latest}
\end{enumerate}
\end{processbox}

\section{Distribution manuelle}

Pour distribuer l'APK sans passer par le Play Store :

\begin{itemize}
    \item \faShare\ \textbf{Partage direct} : Envoyez le fichier \tech{app-release.apk} par email, Bluetooth, WhatsApp ou cl\'e USB.
    \item \faQrcode\ \textbf{QR Code} : H\'ebergez l'APK sur un serveur web et g\'en\'erez un QR code pointant vers l'URL de t\'el\'echargement.
    \item \faCloud\ \textbf{Cloud} : Uploadez l'APK sur Google Drive, Dropbox ou un service \'equivalent et partagez le lien.
\end{itemize}

\section{Publication sur le Google Play Store (optionnel)}

\begin{alertbox}
\textbf{\faExclamationTriangle\ Pr\'erequis :} Un compte Google Play Developer (frais uniques de 25\$) et l'APK sign\'e avec un keystore de production.
\end{alertbox}

\begin{enumerate}
    \item Cr\'eez une fiche d'application sur la \concept{Google Play Console}.
    \item Configurez les informations de l'application (description, captures d'\'ecran, ic\^one).
    \item Uploadez l'APK sign\'e dans la section \concept{Production}.
    \item Soumettez pour examen par Google (d\'elai : 1 \`a 7 jours).
\end{enumerate}


% ════════════════════════════════════════════════════════════════
%     CHAPITRE 7 --- D\'EPANNAGE CI/CD
% ════════════════════════════════════════════════════════════════

\chapter{D\'epannage CI/CD}

\section{Erreurs fr\'equentes}

\begin{table}[H]
\centering
\rowcolors{2}{harmonyGray100}{harmonyWhite}
\begin{tabularx}{\textwidth}{>{\sffamily\small}p{3.2cm} >{\sffamily\small}X >{\sffamily\small}X}
\toprule
\textbf{\textcolor{harmonyPrimary}{Erreur}} & \textbf{\textcolor{harmonyPrimary}{Cause}} & \textbf{\textcolor{harmonyPrimary}{Solution}} \\
\midrule
NDK not found & Version du NDK non install\'ee ou incompatible & V\'erifiez la version dans le workflow : \tech{ndk;27.0.12077973} \\
\addlinespace
Gradle build failed & Version Java incompatible & Utilisez Java 17 (Temurin) comme dans le workflow \\
\addlinespace
Flutter version mismatch & SDK local diff\'erent du CI & Alignez la version dans \tech{flutter-version} du workflow \\
\addlinespace
Keystore not found & Secret mal configur\'e ou \'etape manquante & V\'erifiez les noms des secrets et l'\'etape \tech{Decode Keystore} \\
\addlinespace
Cache miss & Premi\`ere ex\'ecution ou cache expir\'e & Normal au premier run, les suivants seront plus rapides \\
\addlinespace
pub get failed & D\'ependance introuvable ou version incompatible & V\'erifiez \tech{pubspec.yaml} et les contraintes de version \\
\bottomrule
\end{tabularx}
\caption{Erreurs CI/CD fr\'equentes et solutions.}
\label{tab:depannage-cicd}
\end{table}

\section{Lecture des logs GitHub Actions}

\begin{enumerate}
    \item Acc\'edez \`a l'onglet \concept{Actions} du d\'ep\^ot.
    \item Cliquez sur le run \'echou\'e (\statusno\ en rouge).
    \item D\'epliez l'\'etape en \'echec pour voir les logs d\'etaill\'es.
    \item Cherchez les lignes marqu\'ees \textbf{Error} ou \textbf{FAILURE} pour identifier le probl\`eme.
\end{enumerate}

\begin{infobox}
\textbf{\faInfoCircle\ Astuce :} Utilisez la fonction \concept{Search logs} (loupe) en haut des logs pour rechercher rapidement un mot-cl\'e d'erreur sp\'ecifique.
\end{infobox}

\section{Optimisation des temps de build}

\begin{processbox}
\textbf{\textcolor{harmonyPrimary}{\faTachometerAlt\ Strat\'egies d'optimisation :}}

\begin{tabularx}{\textwidth}{>{\bfseries\sffamily}l X >{\sffamily}c}
\toprule
Strat\'egie & Description & Gain estim\'e \\
\midrule
Cache Flutter & D\'ej\`a activ\'e via \tech{cache: true} dans le workflow & $\sim$2 min \\
Cache Gradle & Ajouter \tech{actions/cache} pour \tech{\textasciitilde/.gradle} & $\sim$1 min \\
Filtrer les branches & Ne builder que \tech{main} et \tech{develop} & \'Evite les builds inutiles \\
Skip CI & Ajouter \tech{[skip ci]} dans le message de commit & Build non d\'eclench\'e \\
\bottomrule
\end{tabularx}
\end{processbox}

\begin{codebox}
\begin{lstlisting}
      - name: Cache Gradle
        uses: actions/cache@v4
        with:
          path: |
            ~/.gradle/caches
            ~/.gradle/wrapper
          key: gradle-${{ hashFiles(
            '**/*.gradle*') }}
          restore-keys: gradle-
\end{lstlisting}
\end{codebox}

\elegantsep

\begin{successbox}
\statusok\ \textbf{F\'elicitations !} Votre pipeline CI/CD est d\'esormais pleinement op\'erationnel. Chaque push sur \tech{main} produit automatiquement un APK sign\'e, test\'e et disponible en t\'el\'echargement via GitHub Actions ou GitHub Releases.
\end{successbox}
